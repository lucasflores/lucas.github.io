\chapter[Baseline Object Selection][Baseline Object Selection]{Baseline Object Selection}
\label{app:baselineobjectselection}

%\begin{table}[h]
%  \begin{center}
%  \begin{tabular}{|r|r|}
%  \toprule
%    Electron Triggers & Muon Triggers \\
%  \midrule
%  \multicolumn{2}{|c|}{2015} \\
%  \midrule
%    \texttt{HLT\_e24\_lhmedium\_L1EM20VH} & \texttt{HLT\_mu20\_iloose\_L1MU15} \\
%    \texttt{HLT\_e60\_lhmedium} & \texttt{HLT\_mu50} \\
%    \texttt{HLT\_e120\_lhloose} & \\
%  \midrule
%  \multicolumn{2}{|c|}{2016, 2017, 2018} \\
%  \midrule
%   \texttt{HLT\_e26\_lhtight\_nod0\_ivarloose} & \texttt{HLT\_mu26\_ivarmedium} \\
%   \texttt{HLT\_e60\_lhmedium\_nod0} & \texttt{HLT\_mu50} \\
%   \texttt{HLT\_e140\_lhloose\_nod0} & \\
% \bottomrule
% \end{tabular}
% \end{center}
% \caption{Electron and muon triggers used for each year.}
%\label{tab:triggerSelection}
%\end{table}

\subsection{Electron, Muon, and Jet Selection Summary Tables}
\begin{table}[h]
  \begin{center}
  \begin{tabular}{lc}
  \toprule
    \multicolumn{2}{c}{Baseline electron} \\
  \hline
    Acceptance & \pT$>$10~\GeV, \(|\eta|<2.47\), crack veto \\
    Impact parameter & \(z_0\sin (\theta)\leq 0.5\)~mm \\
    Identification WP & \texttt{LooseAndBLayerLLH} \\
    Object quality & \texttt{BADCLUSELECTRON} electron veto \\
  \hline
    \multicolumn{2}{c}{Signal electron} \\
  \hline
    Acceptance & \pT$>$12~\GeV \\
    Impact parameter &  \(d_0/\sigma(d_0)<5\) \\
    Identification WP & \texttt{MediumLLH} \\
    Isolation WP & \texttt{FCTight} \\
  \bottomrule
  \end{tabular}
  \end{center}
  \caption{Summary of electron selection criteria. Signal criteria are applied on top of baseline criteria after overlap removal.}
  \label{tab:eventSelection:elSelection}
\end{table}

\begin{table}[hp]
  \begin{center}
  \begin{tabular}{lc}
  \toprule
    \multicolumn{2}{c}{Baseline muon} \\
  \hline
    Acceptance & \pT$>$10~GeV, \(|\eta|<2.7\) \\
    Impact parameter & \(z_0\sin (\theta)\leq 0.5\)~mm \\
    Identification WP & \texttt{Medium} \\
  \hline
    \multicolumn{2}{c}{Signal muon} \\
  \hline
    Acceptance & \pT$>$12~\GeV \\
    Impact parameter &  \(d_0/\sigma(d_0)<3\) \\
    Identification WP & \texttt{Medium} \\
    Isolation WP & \texttt{FCTight\_FixedRad} \\
    Object quality & Cosmic muon veto, bad muon event veto \\
  \bottomrule
  \end{tabular}
  \end{center}
\caption{Summary of muon selection criteria. Signal criteria are applied on top of baseline criteria after overlap removal.}
\label{tab:eventSelection:muSelection}
\end{table}

\begin{table}[hp]
  \begin{center}
  \begin{tabular}{lc}
  \toprule
    \multicolumn{2}{c}{Baseline jet} \\
  \hline
    Acceptance & \pT$>$20~\GeV \\
  \hline
    \multicolumn{2}{c}{Signal jet} \\
  \hline
    Acceptance & \(|\eta|\)$<$2.8 \\
    JVT & Medium (\pT$<$120~\GeV, \(|\eta|<\)2.5) \\
    Object quality & \texttt{LooseBad} event veto \\
  \hline
    \multicolumn{2}{c}{Signal b-jet} \\
  \hline
    Acceptance & \(|\eta|\)$<$2.5 \\
    b-tagger algorithm & \texttt{MV2c10} \\
    b-tagging WP & \texttt{Fixed 85\%} \\
  \bottomrule
  \end{tabular}
  \end{center}
  \caption{Summary of jet selection criteria. Signal criteria are applied on top of baseline criteria after overlap removal.}
  \label{tab:eventSelection:jetSelection}
\end{table}
\newpage
\subsection{Photon Selection}
While photons are not used as signal objects in this analysis,
they are used as input to the missing energy calculation.
Photons have a minimum transverse momentum of 25~\GeV and a maximum \(|\eta|\) of 2.37, with a veto for photons that fall in the calorimeter crack region, corresponding to \(|\eta|\) between 1.37 and 1.52.
The \texttt{BADCLUSPHOTON} object quality criteria is required, as well as an author selection.

\subsection{Missing energy}
The missing transverse energy is calculated with the \texttt{Tight} working point,
using all calibrated baseline objects in the event (electrons, muons, jets, and photons) as well as all tracks matched to the primary vertex not associated with these objects.
Baseline jets are used only if they are tagged as originating from the hard scatter, using JVT.
The \met\ significance is also calculated at this stage using Asg tools.

\subsection{Overlap removal}
\label{sec:overlapRemoval}
Overlap removal is performed using the Asg tool and the standard recommendations therein,
as outlined in \href{https://indico.cern.ch/event/631313/contributions/2683959/}{https://indico.cern.ch/event/631313/contributions/2683959/}, with the alteration that the $b$-jet overlap removal is \pT-dependent as described in the steps below
Technically, this is achieved by creating a standalone collection of jets which are only $b$-tagged in the desired \pT\ range.
This collection of $b$-jets, rather than the standard collection used in the rest of the analysis, is then passed to the Asg tool.
All objects that are rejected by overlap removal are removed from further overlap removal steps, and from future consideration in the analysis.
Overlap removal follows the steps in the order outlined here:
\begin{itemize}
  \item{Electrons that share a track with another higher-\pT\ electron are rejected.}
  \item{Electrons that share a track with a non-calorimeter-tagged muon are rejected.}
  \item{Jets that are not $b$-tagged, or that are b-tagged with \pT$>$100~\GeV, and are within \(\Delta R(e,\mathrm{jet})\leq0.2\) of an electron are rejected.}
  \item{Electrons that are within \(\Delta R(e,\mathrm{jet})\leq0.4\) of a jet are rejected.}
  \item{Jets that are not $b$-tagged, or that are $b$-tagged with \pT$>$100~\GeV, and are ghost-matched to a muon (or within \(\Delta R(\mu,\mathrm{jet})\leq0.2\)) and which satisfies \(n_{\mathrm{track}}<3\) are rejected.}
  \item{Muons that are within \(\Delta R(\mu,\mathrm{jet})\leq0.4\) of a jet are rejected.}
\end{itemize}