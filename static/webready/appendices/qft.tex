
\chapter[Formulation of QFT][Formulation of QFT]{Formulation of QFT}
\section{Formulation of QFT}\label{app:qft}
I find it easiest, and maybe the most logically satisfying, to think of the formulation of QFT in terms of canonical quantization (or second quantization). 
Whereby the idea here is to retain the familiar form of the classical Hamiltonian (or Langrangian) representation and then leap from a classical theory to a quantum one by promoting fundamental measurables of physical objects to operators and Poisson brackets to commutators.
So more explicitly the steps are 
\begin{enumerate}
    \item Assume the quantum field Hamiltonian density has the same form as the classical field Hamiltonian density
    \item Replace the classical Poisson brackets for conjugate property densities with commutator brackets (divided by i$\hbar$), i.e.
    $A\rightarrow \hat{A}, \hspace{0.2cm} \{A,B\} \rightarrow\frac{1}{i\hbar}[\hat{A},\hat{B}]$, where the ``hatting'' of the variables signifies the the classical field dynamical variables becoming quantum field non-commuting operators as a consequence of this imposition.
\end{enumerate}
As an example we wlll quantize the most basic resulting QFT, scalar field theory. 
The classical scalar field, $\phi(x,t)$, takes in the position and time and produces the value of the field at that position and time.
Classically, a scalar field is a collection of an infinity of oscillator normal modes.
So the classical Lagrangian density describing an infinite number of coupled harmonic oscillators is written as
\begin{equation}
    \mathcal{L}(\phi)= \frac{1}{2}(\partial_{t}\phi)^2 - \frac{1}{2}(\partial_{x}\phi)^2 - \frac{1}{2}m^{2}\phi^{2} - V(\phi)
\end{equation}
Where $V(\phi)$ is a potential term.
The action is then,
\begin{equation*}
    S(\phi)= \int \mathcal{L}(\phi)dxdt = \int L(\phi, \partial_{t}\phi)dt
\end{equation*}
The canonical momentum can be obtained from the action via the Legendre transformation, and is is found to be  $\displaystyle \pi =\partial _{t}\phi$.
The classical Hamiltonian is then,
\begin{equation}
    H(\phi,\pi) = \int dx\left[ \frac{1}{2}\pi^2 + \frac{1}{2}(\partial_{x}\phi)^2 + \frac{1}{2}m^{2}\phi^{2} + V(\phi) \right]
    \label{eq:theory:hammy}
\end{equation}
Next we impose the canonical commutation relations at $t$=0 as follows
\begin{equation*}
 [\phi(x),\phi(y)] = 0,  \hspace{0.5cm} [\pi(x), \pi(y)] = 0, \hspace{0.5cm} [\phi(x),\pi(y)] = i\hbar \delta(x-y)
\end{equation*}
The operators can then be generalized to and time $t$ in the future by applying the time evolution operator $\mathcal{O}$,
\begin{equation*}
     \mathcal{O}(t) = e^{itH} \mathcal{O} e^{-itH} .
\end{equation*}
Where at this point a choice of $V(\phi)$ is required.
For simplicity we will just consider the case of the free field with $V(\phi)$=0.
It is useful to Fourier transform the fields,
\begin{equation*}
      \phi_k = \int \phi(x) e^{-ikx} dx, \hspace{0.5cm} \pi_k = \int \pi(x) e^{-ikx} dx .
\end{equation*}
It can then be identified that,
\begin{equation*}
    \phi_{-k} = \phi_k^\dagger, \hspace{0.5cm} \pi_{-k} = \pi_k^\dagger.
\end{equation*}
Expanding the Hamiltonian density in Equation~\ref{eq:theory:hammy} in Fourier modes,
\begin{equation}
     H=\frac{1}{2}\sum_{k=-\infty}^{\infty}\left[\pi_k \pi_k^\dagger + \omega_k^2\phi_k\phi_k^\dagger\right],
     \label{eq:theory:quantumhammy}
\end{equation}
where $ {\displaystyle \omega _{k}={\sqrt {k^{2}+m^{2}}}}\omega_k = \sqrt{k^2+m^2}$.
We recognize this Hamiltonian as an infinite sum of classical oscillators $\phi_{k}$, each one of which is quantized in the standard manner, so the free quantum Hamiltonian looks identical. It is the $\phi_{k}$s that have become operators obeying the standard commutation relations, $[\phi_{k},\pi_{k}^{\dagger}] = [\pi_{k}^{\dagger},\phi_{k}] = i\hbar$  with all others vanishing.
The Hilbert space of all these oscillators is constructed using creation and annihilation operators determined from these modes,
\begin{equation*}
     a_k = \frac{1}{\sqrt{2\hbar\omega_k}}\left(\omega_k\phi_k + i\pi_k\right), \ \ a_k^\dagger = \frac{1}{\sqrt{2\hbar\omega_k}}\left(\omega_k\phi_k^\dagger - i\pi_k^\dagger\right), 
\end{equation*}
Subtracting of the zero-point energy $\hbar\omega_{k}/2$ from the Hamiltonian in Equation~\ref{eq:theory:quantumhammy} in order to satisfy the condition that $H$ must annihilate the vacuum and rewriting in terms of the creation/annihilation operators teh Hamiltonian takes the form,
\begin{equation}
     H = \sum_{k=-\infty}^{\infty} \hbar\omega_k a_k^\dagger a_k = \sum_{k=-\infty}^{\infty} \hbar\omega_k N_k
\end{equation}
where $N_k$ may be interpreted as the number operator giving the number of particles in a state with momentum $k$.
% Would like to put a few example Lagrangians of other theories (phi 4 , complex scalar,, and ) with brief qualitative descriptions 
Now, commutation relations are useful only for quantizing bosons, for which the occupancy number of any state is unlimited. To quantize fermions, which satisfy the Pauli exclusion principle, anti-commutators are needed. i.e. the relation  \{A,B\} = AB+BA.
When quantizing fermions, the fields are expanded into the creation and annihilation operators, $b_{k}^{\dagger}$, $b_{k}$, which satisfy
\begin{equation}
    \{ b_{k}, b_{l}^{\dagger}\} = \delta_{k,l}, ~  \{ b_{k}, b_{l}\} = 0, ~ \{ b_{k}^{\dagger}, b_{l}^{\dagger}\} = 0
\end{equation}

All other fields can be quantized by a generalization of this procedure. Vector or tensor fields simply have more components, and independent creation and destruction operators must be introduced for each independent component. If a field has any internal symmetry, then creation and destruction operators must be introduced for each component of the field related to this symmetry as well. If there is a gauge symmetry, then the number of independent components of the field must be carefully analyzed to avoid over-counting equivalent configurations, and gauge-fixing may be applied if needed~\cite{2013.Klauber}.



%begin{equation}
%Q_{\{f,g\}}=\frac{1}{i\hbar}[Q_{f},Q_{g}]
%\label{eq:qft}
%\end{equation}
