\chapter[Introduction][Introduction]{Introduction}
%\textcolor{red}{\hrulefill \textsc{Unfinished Section}\hrulefill}\\
The field of elementary particle physics has been a burgeoning space of discovery since the development of particle accelerators in the 1950s.
As accelerator technology advanced allowing for higher energy collisions more and more new particles were being discovered, contributing to what became known as the ``particle zoo.''
Fortunately this zoo was reined in with completion of the theory called the Standard Model in the 1970s.
This theory reduced the number of actual elementary particles to a handful, filing most of the zoo into composite particles made from the elementary particles. 
Since its emergence in the 1970's the SM has been remarkably successful as particle accelerator experiments have verified its predictions to great precision.
And with the discovery of the Higgs boson by the ATLAS and CMS experiments at the \gls{cern} in 2012 the SM in its modern form was completed.
However we know that that cannot be the entire story.
Observed unexplained phenomena such as dark matter and unanswered question such as why the Higgs mass is so light demand there be physics beyond the Standard Model.
And it is in this thesis where one approach for searching for new physics is described.
This thesis is divided into four main chapters and is organised as follows. 
Chapter~\ref{ch:theory} describes the theoretical framework of the SM as well as an extension to that framework that would allow for new fundamental particles that would be accessible at high energy particle physics experiments such as the Large Hadron Collider (LHC) and its detectors.
Chapter~\ref{ch:detector} then describes the LHC and its beam, ATLAS and its constituent subdetectors, and ATLAS's methodology for reconstructing collision events and collecting data.
Chapter~\ref{ch:electronid} details the full procedure for reconstructing and identifying electrons at ATLAS and my contributions to that effort.
And finally Chapter~\ref{ch:rpvthreel} describes a beyond the Standard Model search for a new fundamental particle via its decay into three charged leptons (electrons of muons) as motivated by an $R$-Parity violating supersymmetric model.


%The \gls{sm}\footnotemark~has been remarkably successful...

%\footnotetext{Here's a footnote.}
%\gls{atlas}
%\gls{photon}
%\gls{qft}
%\rp 