\chapter{Preface}
Since at least high school I have been enamored with ``\emph{big questions}.''
And particle physics was where I found many of those questions were trying to be answered, endeavoring to probe the smallest and most fundamental constituents of reality.
The path then sort of took shape itself, as (for me) there was really only one game in town for experimental fundamental particle physics, the Large Hadron Collider (LHC).
This ultimately led me to the University of Pennsylvania (Penn) to pursue a Ph.D. in experimental high energy particle physics on the ATLAS experiment (one of two general purpose detectors on the LHC).
I started at Penn in 2015, which was an exciting time at the LHC as it was the end of a 2-year technical stop that prepared the machine for running at almost double the energy of the LHC’s first run.
The then CERN Director-General Rolf Heuer summarizes the excitement in the air prefacing the start of Run 2:
\begin{quote}
    ``\emph{With this new energy level, the LHC will open new horizons for physics and for future discoveries}''
\end{quote}
And I was lucky enough to take my first trip out to CERN the summer of 2015 for a few weeks, certainly a memorable trip as I lost my wallet almost immediately (I survived though).
%But nevertheless I was at CERN, a historical laboratory where the world wide web was invented and the Higgs boson discovered.

The next two years would be spent living in Philadelphia, taking classes at Penn, and working on electron identification as a part of the electron/photon ($e/\gamma$) performance group on ATLAS.  
Initially I investigated a new variable that utilizes the occupancy of the transition radiation tracker (a sub-detector on ATLAS responsible for tracking particle trajectories) to better represent the noisy activity near an electron candidate in order for our electron identification (a likelihood based method) to remain efficient in very busy environments.
It then became important to pivot to a re-optimization of the identification when enough Run 2 data became available, allowing the identification algorithm to become fully data-driven for the remainder of Run 2.
This was ultimately the work that led to my authorship on ATLAS.
%Additionally I was responsible for the maintenance of the electron identification codebase (framework) 

After several years spent working with some wonderful people in the $e/\gamma$ group I then made the transition to analysis work where I joined the fully Penn-driven team, led by my advisor Evelyn Thomson, to look for new physics in the context of an $R$-parity violating supersymmetric model.
Whereas my luck would continue and I would get to work with even more wonderful people doing interesting physics.
Here I would work on developing regions where major backgrounds could be estimated, 
It was also around this time that I moved from Philadelphia out to CERN (Geneva, Switzerland more specifically) for what was then an unknown amount of time, but where I would ultimately spend the remainder of my Ph.D.
Being able to be on-the-ground at CERN and live in the beautiful city of Geneva was an unforgettable experience where I met some life long friends.


\vspace{0.05\textheight}

% Feel free to format this however you like, if you want it at all.
\begin{tabular}{p{0.5\textwidth} l}
  & Lucas Macrorie Flores             \\
  & Earth, September 2021  \\
\end{tabular}

