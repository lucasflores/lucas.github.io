%\textcolor{red}{\hrulefill \textsc{Unfinished Section}\hrulefill}\\
The topo-cluster reconstruction algorithm begins by forming proto-clusters in the EM and hadronic calorimeters using a set of noise thresholds in which the cell initiating the cluster is required to have $|\zeta_{\mathrm{cell}}^{\mathrm{EM}}|\geq$ 4, where
\begin{equation}
    \zeta_{\mathrm{cell}}^{\mathrm{EM}}= \frac{E_{cell}^{\mathrm{EM}}}{\sigma_{\mathrm{noise,cell}}^{\mathrm{EM}}}
\end{equation}
$E_{cell}^{\mathrm{EM}}$ is the cell energy at the EM scale and $\sigma_{\mathrm{noise,cell}}^{\mathrm{EM}}$ is the expected cell noise~\cite{EGAM-2018-01}.
The expected cell noise includes the known electronic noise and an estimate of the pile-up noise corresponding to the average instantaneous luminosity expected.
In this initial stage, cells from the presampler and the first LAr EM calorimeter layer are excluded from initiating proto-clusters, to suppress the formation of noise clusters.
The proto-clusters then collect neighboring cells with significance $|\zeta_{\mathrm{cell}}^{\mathrm{EM}}|\geq$ 2.
Each neighbor cell passing the threshold of $|\zeta_{\mathrm{cell}}^{\mathrm{EM}}|\geq$ 2 becomes a seed cell in the next iteration, collecting each of its neighbors in the proto-cluster.
If two proto-clusters contain the same cell with $|\zeta_{\mathrm{cell}}^{\mathrm{EM}}|\geq$ 2 above the noise threshold, these proto-clusters are merged.
A crown of nearest-neighbor cells is added to the cluster independently on their energy.
%In the presence of negative-energy cells induced by the calorimeter noise, the algorithm uses $ $ instead of $ $ to avoid biasing the cluster energy upwards, which would happen if only positive-energy cells were used.
This set of thresholds is commonly known as ‘4-2-0’ topo-cluster reconstruction.
Proto-clusters with two or more local maxima are split into separate clusters; a cell is considered a local maximum when it has $E_{cell}^{\mathrm{EM}}>$ 500 \MeV, at least four neighbors, and when none of the neighbors has a larger signal.