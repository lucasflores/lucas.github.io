To continue any further we must detail one of the most powerful, and the most essential, tools that an experimental particle physicist has at their disposal, simulated data by way of Monte Carlo (MC) experiments.
Monte Carlo simulation is used to model the expected contributions of various SM processes as well the \CCsignal and \CNsignal signal processes targeted by this search. 
The standard model MC samples form the basis of what will ultimately be the null hypothesis of the statistical test that will be used to evaluate against the data whether or not a discovery can be claimed. 
It also functions as a crucial tool to define and optimize event selection criteria and to estimate systematic uncertainties in the event yield predictions.
MC simulation has three stages: First, event generation for each physics process (e.g. $pp \rightarrow$\Wboson\Zboson$\rightarrow \ell\nu~\ell\ell$) to predict the momenta of the final state particles; second, detector simulation to predict the response of the detector to these particles in terms of signals in the charged particle tracking detectors and energy deposits in the calorimeters; and third, the application of the same detector reconstruction algorithms used for the actual data to form the physics objects (electrons, muons, photons, jets, and missing transverse energy).

\subsection{MC Specifications}
The generators and parameters used in the MC simulation samples for this analysis are summarized in Table \ref{tab:MC}.
Signal samples were generated at masses between 100~\GeV and 1500~\GeV in steps of 50~\GeV.
Signals with masses below 100~\GeV were not explored as they have been excluded by previous searches for charginos and neutralinos~\cite{SUSY-2016-24,SUSY-2013-12,CMS-SUS-16-039,CMS-SUS-17-004,EXOT-2014-07,EXOT-2014-08,CMS-EXO-17-006}.
Signal events were generated with equal \chono branching fractions to each boson ($W$, $Z$, or 
Higgs bosons where kinematically accessible) plus a lepton ($e$, $\nu_{e}$, $\mu$, $\nu_{\mu}$, $\tau$-lepton, or $\nu_{\tau}$) channel.
In order to explore different assumptions for the \chono branching fractions in the analysis, simulated events are reweighted appropriately, assuming that the \chone and \none branching fractions change in the same way.
Generated signal events were required to have at least three leptons, two of which were associated with a $Z$ boson.
Hadronically decaying $\tau$-leptons were not considered by this three-lepton requirement for the \CNsignal events.
The \chone were also required to decay via a $Z$ boson in the \CNsignal events to increase the number of events with a trilepton resonance.
The inclusive production cross sections were calculated assuming mass-degenerate, wino-like \chone and \none, as predicted by the \BL RPV model~\cite{Dumitru:2018jyb}, and were calculated at NLO in QCD with next-to-leading-logarithmic (NLL) corrections to the soft-gluon terms~\cite{Beenakker:1999xh,Debove:2010kf,Fuks:2012qx,Fuks:2013vua,Fiaschi:2018hgm}.
The cross sections and their uncertainties were derived from an envelope of cross-section predictions using different PDF sets and factorization and renormalization scales~\cite{Borschensky:2014cia}.
The inclusive cross sections for \CCsignal (\CNsignal) production at a center-of-mass energy of \rts\ =13~\TeV range from $11.6\pm0.5~(22.7\pm1.0)$~pb for masses of 100~\GeV to $0.040\pm0.006~(0.080\pm0.013)$~fb for masses of 1500~\GeV.
Events from all generators were propagated through a full simulation of the ATLAS detector~\cite{SOFT-2010-01} using \textsc{Geant4}~\cite{Agostinelli:2002hh}
to model the interactions of particles with the detector.
A parameterized simulation of the ATLAS calorimeter~\cite{SOFT-2010-01} was used for faster detector simulation of signal, \tW, and \ttH processes and was found to be in agreement with the full simulation.
The effect of multiple interactions in the same and neighboring bunch crossings (pileup) was modeled by overlaying simulated minimum-bias events onto each hard-scattering event.


\begin{table}[t]
\footnotesize
\centering{
\caption[Details of the MC simulation for each physics process]{
Details of the MC simulation for each physics process, including the event generator used for matrix element calculation,
the generator used for the PS and hadronization, the PS parameter tunes, and the order in \alphas~ of the production cross-section calculations.
}
\label{tab:MC}
  \begin{tabular*}{\textwidth}{l @{\extracolsep{\fill}} c  @{\extracolsep{\fill}} c c  @{\extracolsep{\fill}} c}
  \toprule
    \multirow{2}{*}{Process} & \multirow{2}{*}{Event generator}    & PS and        & \multirow{2}{*}{PS tune} & \multirow{2}{*}{Cross section (in QCD)} \\
                             &                                     & hadronization &                          & \\
    \hline
    \hline

    Diboson, triboson, (\Zjet)  & \SHERPAV{2.2}                 & \SHERPAV{2.2} & Default & \scriptsize{NLO (NNLO)} \\
    \ttW, \ttZ, (Other top)     & \scriptsize{\MGMCatNLOV{2}}   & \PYTHIAV{8}   & A14     & \scriptsize{NLO (LO)} \\
    \ttbar, (\tW), [\ttH{}]     & \POWHEGBOX~v2                 & \PYTHIAV{8}   & A14     & \scriptsize{NNLO+NNLL (NLO+NNLL) [NLO]} \\
    Higgs: ggF, (VBF, $VH$)     & \POWHEGBOX~v2                 & \PYTHIAV{8}   & AZNLO  & \scriptsize{NNNLO (NNLO+NNLL)} \\
    \CCsignal, \CNsignal        & \MADGRAPHV{2.6}               & \PYTHIAV{8}   &  A14    & \scriptsize{NLO+NLL} \\

  \bottomrule
  \end{tabular*}
}
\end{table}