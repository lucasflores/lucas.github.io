%% \chapter[htoc-titlei][hhead-titlei]{htitlei}
%% -----------------------------------------------------------------------------
\chapter[Conclusion][Conclusion]{Conclusion}

In this thesis we have discussed the theoretical framework that underpins our current understanding of elementary particle physics, the Standard Model, in Chapter~\ref{ch:theory}.
We also proposed the \BL MSSM as a potential extension to the SM that provides a more natural way for $R$-parity conservation to be handled.
The breaking of the $U(1)_{B-L}$ symmetry gives rise to $R$-parity violating couplings that allow for many interesting final states.
Then the fantastical LHC device and the general purpose ATLAS detector which are well equipped to look for these particle signatures were described in Chapter~\ref{ch:detector}.
Much of the work in the $e$/$\gamma$ performance group, including a detailed description of the electron likelihood identification technique and my contributions  optimizing and developing working points used by the majority of ATLAS analyses, was detailed in Chapter~\ref{ch:electronid}.
Then in the context of the \BL MSSM we motivated \chone and \none as likely LSPs.
The signal \triLepDecay via \CCsignal and \CNsignal productions was searched for in 139~\ifb of proton-proton data collected at ATLAS.
Search methodology and results of this analysis were presented in Chapter~\ref{ch:rpvthreel} and no statistically significant excess was seen.
Limits were then set in the Mass vs. Branching Ratio to \Zboson\ and Branching Ratio to \Zboson\  vs. Branching Ratio to \Hboson\ planes, scanned over four lepton flavor branching ratio points. 
%While no new physics was discovered...

%\paragraph*{Looking Ahead} \hspace{0pt} \\

%Here's an example of how to have an ``informal subsection''.
