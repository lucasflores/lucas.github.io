We won't go as far back as classical mechanics but there is good reason to take a step back to what Quantum Field Theory (QFT) \emph{is}, as all of our current understanding of elementary particle physics is in the context of quantum field theories. 
That being said, there is no strict canonical definition of what QFT actually \emph{is}, which might be why you don't see a section like this often in similar theses.
Nevertheless we can gain enough handles in discussing it in relation to other physical theories.
A common characterization is to sum up QFT as being the reconciliation of quantum mechanics (QM) with special relativity, which while true does not give us the full picture, as relativistic QM exists in the form of the Klein Gordon and Dirac equations, and it is also possible to form a non-relativistic QFT as well~\cite{Bain:2011}.
A potentially more discerning description would be that QFT, and not QM, allows for the description of systems with an infinite amount of degrees of freedom, i.e. fields.
Of course this alone also falls just short as a classical field theory (relativistic or non-relativistic) is equipped for such a task as well.
So the marriage of quantization, relativity, and field theory is all necessary in order to capture what makes QFT QFT.
The Appendix~\ref{app:qft} has a brief review of the formulation of QFT and may be useful for readers before moving into the next section.
%The need for what eventually became \gls{qft}, was to reconcile the extremely successful quantum mechanics with special relativity.
%Mixing in classical field theory and out pops out \gls{qft}
%And with the advent of the Schrodinger wave mechanics formulation of quantum mechanics Dirac was able to give the world its first successful \gls{qft}, \gls{qed}. 

%maybe move to Aux material
