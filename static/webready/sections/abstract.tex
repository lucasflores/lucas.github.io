%This is the abstract text. Note that this is not allowed to exceed 350 words.
In this thesis a search for particles in the context of the theory of supersymmetry (SUSY) as well as algorithms for reconstructing and identifying electrons are detailed.
The physics search was performed using 139~\ifb\ of $\sqrt{s}=13$ \tev proton-proton collision data produced by the Large Hadron Collider (LHC) and collected by the ATLAS experiment during data taking periods in 2016, 2017, and 2018.
Bare SUSY gives rise to baryon ($B$) and lepton ($L$) number violation which would lead to rapid proton decay and is therefore incomplete as we do not see this in nature. 
The ``\BL\ MSSM" solves this elegantly by introducing a new local symmetry, $U(1)_{B-L}$, which accounts for the apparent strict conservation of both B and L at low energies as well as provides a method for generating neutrino masses and breaking SUSY.  
This model allows for the lightest supersymmetric particle (LSP) to decay into Standard Model particles giving rise to a rich phenomenology. 
One such signature is the decay of the SUSY partner of the \Wboson\ and charged Higgs bosons, \chone, into a final state with three charged leptons.
This is a clean signature in which the invariant mass of the particle can be fully reconstructed. 
The search for this signature is the primary focus of this thesis.
The second focus of this thesis is the developments and optimizations of the algorithms and software pertaining to the reconstruction and identification of electrons.  
 
